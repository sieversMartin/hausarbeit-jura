% \iffalse meta-comment
%<*internal>
\iffalse
%</internal>
%<*internal>
\fi
\def\nameofplainTeX{plain}
\ifx\fmtname\nameofplainTeX\else
  \expandafter\begingroup
\fi
%</internal>
%<*install>
\input docstrip.tex
\keepsilent
\askforoverwritefalse
\preamble
----------------------------------------------------------------
hausarbeit-jura -- A LaTeX class for writing "juristische 
                   Hausarbeiten" at German universities.
E-Mail: martin.sievers@schoenerpublizieren.de
Released under the LaTeX Project Public License v1.3c or later
See http://www.latex-project.org/lppl.txt
----------------------------------------------------------------

\endpreamble
\postamble

Originally developed by Martin Sievers (martin.sievers@schoenerpublizieren.de)
Copyright (C) 2012-2016 by Martin Sievers

This work may be distributed and/or modified under the
conditions of the LaTeX Project Public License (LPPL), either
version 1.3c of this license or (at your option) any later
version. The latest version of this license is in the file:

 http://www.latex-project.org/lppl.txt

This work is "maintained" (as per LPPL maintenance status) by
  Martin Sievers (martin.sievers@schoenerpublizieren.de).

This work consists of the file  hausarbeit-jura.dtx
                                hausarbeit-jura.ins
                                README.md
          and the derived files hausarbeit-jura.pdf
                                hausarbeit-jura.cls
                                hausarbeit-demo.tex
                                hausarbeit-demo.bib
                                
\endpostamble
\usedir{tex/latex/hausarbeit-jura}
\generate{
  \file{\jobname.cls}{\from{\jobname.dtx}{class}}
}

\nopreamble\nopostamble
\usedir{doc/latex/hausarbeit-jura}
\generate{
  \file{hausarbeit-demo.tex}{\from{\jobname.dtx}{demo}}
  \file{hausarbeit-demo.bib}{\from{\jobname.dtx}{bib}}
}
%</install>
%<install>\endbatchfile
%<*internal>
\usedir{source/latex/hausarbeit-jura}
\generate{
  \file{\jobname.ins}{\from{\jobname.dtx}{install}}
}
\ifx\fmtname\nameofplainTeX
  \expandafter\endbatchfile
\else
  \expandafter\endgroup
\fi
%</internal>
%<*driver>
\ProvidesFile{hausarbeit-jura.dtx}%
[2016/09/10 v1.5 Eine Klasse für juristische Hausarbeiten]
\documentclass[a4paper]{ltxdoc}
\usepackage[english,ngerman]{babel}
\usepackage[utf8]{inputenc}
\usepackage[T1]{fontenc}
\IfFileExists{libertine.sty}
	{\usepackage{libertine}%
	 \IfFileExists{beramono.sty}
	 	{\usepackage[scaled=0.85]{beramono}}{}
	}
	{\usepackage{lmodern}}
\IfFileExists{microtype.sty}
	{\usepackage{microtype}}{}
\usepackage{textcomp}
\usepackage{upquote}
\usepackage{array,booktabs,tabularx}
\usepackage[final]{listings}
\usepackage[onehalfspacing]{setspace}
\usepackage{xspace}
\usepackage[dvipsnames]{xcolor}
\usepackage[%
	pdftitle={hausarbeit-jura: A LaTeX class for writing 
         	``juristische Hausarbeiten'' at German universities.},
   pdfauthor={Martin Sievers},
   urlcolor=blue,%
	linktocpage,%
	colorlinks=true]{hyperref}
\OnlyDescription     %nur Anleitung (ohne Index und History)
\CodelineIndex       %kein Index wenn auskommentiert
\EnableCrossrefs     %kein Index wenn auskommentiert
\RecordChanges       %keine History wenn auskommentiert
\begin{document}
\DeleteShortVerb{\|}
\DocInput{hausarbeit-jura.dtx}
\end{document}
%</driver>
% \fi
% \CheckSum{0}
% \CharacterTable
%  {Upper-case    \A\B\C\D\E\F\G\H\I\J\K\L\M\N\O\P\Q\R\S\T\U\V\W\X\Y\Z
%   Lower-case    \a\b\c\d\e\f\g\h\i\j\k\l\m\n\o\p\q\r\s\t\u\v\w\x\y\z
%   Digits        \0\1\2\3\4\5\6\7\8\9
%   Exclamation   \!     Double quote  \"     Hash (number) \#
%   Dollar        \$     Percent       \%     Ampersand     \&
%   Acute accent  \'     Left paren    \(     Right paren   \)
%   Asterisk      \*     Plus          \+     Comma         \,
%   Minus         \-     Point         \.     Solidus       \/
%   Colon         \:     Semicolon     \;     Less than     \<
%   Equals        \=     Greater than  \>     Question mark \?
%   Commercial at \@     Left bracket  \[     Backslash     \\
%   Right bracket \]     Circumflex    \^     Underscore    \_
%   Grave accent  \`     Left brace    \{     Vertical bar  \|
%   Right brace   \}     Tilde         \~}
%
% \changes{v1.0}{2012/02/08}{First ``official'' version, still without documentation}
% \changes{v1.0.1}{}{dtx now extracts ``README.txt'' instead of ``README''.\\Code cleaning of the dtx file}
% \changes{v1.1}{2014/05/13}{Added a documentation}
% \changes{v1.1}{2014/05/09}{Modified demo file}
% \changes{v1.2}{2016/04/04}{Rearranged package (not only) for GitHub}
% \changes{v1.2}{2016/04/04}{Added option `headline' (thanks to Tobias Hirning) to add more information to the header}
% \changes{v1.2}{2016/04/04}{Introduced alternative interface for title page: \cs{matrikelnummer} and \string\semester[]{}}
% \changes{v1.2}{2016/04/04}{TeX Gyre Fonts are now the standard fonts; new option ``oldfonts'' for compatibility}
% \changes{v1.2}{2016/04/04}{Added support for XeLaTeX and LuaLaTeX}
% \changes{v1.3}{2016/04/11}{Bugfix: missing \cs{fi} introduced with v1.2}
% \changes{v1.3}{2016/04/11}{Added an option headlinetitlepageleft (thanks to Tobias Hirning) to set the header left-aligned on the titlepage}
% \changes{v1.3}{2016/04/11}{Added package ellipsis}
% \changes{v1.4}{2016/04/24}{Added macros to change paper size used in frontmatter and mainmatter (suggested by Adi Sander)}
% \changes{v1.4}{2016/04/25}{Bugfix: \cs{onehalfspacing} and \cs{reversemarginpar} have to be called after each \cs{loadgeometry}}
% \changes{v1.4}{2016/04/25}{Added definition for \cs{subsubsection}}
% \changes{v1.4}{2016/04/25}{Modifed microtype options}
% \changes{v1.5}{2016/09/10}{Added \cs{sectionbefore} and \cs{sectionafter} to \cs{section}}
%
% \GetFileInfo{\jobname.dtx}
% \DoNotIndex{\newcommand,\newenvironment}
% \DoNotIndex{\def,\edef,\gdef,\xdef,\global,\long,\let}
% \DoNotIndex{\expandafter,\string,\the,\ifx,\else,\fi}
% \DoNotIndex{\csname,\endcsname,\relax,\begingroup,\endgroup}
% \DoNotIndex{\DeclareTextCommand,\DeclareTextCompositeCommand}
% \DoNotIndex{\space,\@empty,\special,\@nil,\advance\@nnil}
% \DoNotIndex{\\,\@gobble,\@@,\@fornoop,\@fortmp,\@ifundefined}
% \DoNotIndex{\@tempcnta,\@tempcntb,\{,\},\alph,\bgroup,\egroup}
% \DoNotIndex{\do,\end,\HN,\ifcase,\ifnum,\IfFileExists,\ifvmode}
% \DoNotIndex{\ignorespaces,\immediate,\input,\item,\jobname}
% \DoNotIndex{\leavevmode,\loop,\repeat,\makeatletter,\makeatother}
% \DoNotIndex{\meaning,\newcounter,\next,\or,\par,\renewcommand}
% \DoNotIndex{\renewcommand,\renewenvironment,\stepcounter}
% \DoNotIndex{\Tg,\thepage,\unskip,\write,\advance,\{,\}}
% \DoNotIndex{\chapter,\subsection,\subsubsection,\subthreesection}
% \makeatletter
% \newcommand*\DescribeOption{^^A
%  \leavevmode
%  \@bsphack
%  \begingroup
%    \MakePrivateLetters
%    \Describe@Option
%}
% \newcommand*\Describe@Option[1]{^^A
%    \endgroup^^A
%  \marginpar{^^A
%    \raggedleft^^A
%    \PrintDescribeEnv{#1}^^A
%  }%
%  \SpecialOptionIndex{#1}^^A
%  \@esphack^^A
%  \ignorespaces^^A
% }
% \newcommand*\SpecialOptionIndex[1]{^^A
%  \@bsphack
%  \index{^^A
%    #1\actualchar{\protect\ttfamily#1} (option)\encapchar usage^^A
%  }^^A
%  \index{^^A
%    options:\levelchar#1\actualchar{\protect\ttfamily#1}
%    \encapchar usage^^A
%  }^^A
%  \@esphack
% }
%
%^^A For creating examples with nice highlighting of code, and so
%^^A on; based on the system used in the listings source (lstsample).
%\lst@RequireAspects{writefile}
%\newsavebox{\LaTeXdemo@box}
%\lstnewenvironment{LaTeXdemo}[1][common]{^^A
%  \global\let\lst@intname\@empty
%  \expandafter\let\expandafter\LaTeXdemo@end
%    \csname LaTeXdemo@#1@end\endcsname
%  \@nameuse{LaTeXdemo@#1}^^A
%}{^^A
%  \LaTeXdemo@end
%}
%\newcommand*\LaTeXdemo@new[3]{^^A
%  \expandafter\newcommand\expandafter*\expandafter
%    {\csname LaTeXdemo@#1\endcsname}{#2}^^A
%  \expandafter\newcommand\expandafter*\expandafter
%    {\csname LaTeXdemo@#1@end\endcsname}{#3}^^A
%}
%\newcommand*\LaTeXdemo@common{^^A
%  \lstset{
%    basicstyle   = \small\ttfamily,
%    ^^A basewidth    = 0.51em,
%    gobble       = 3,
%    keywordstyle = \color{blue}\bfseries,
%    language     = [LaTeX]{TeX},
%    moretexcs    = {,
%      semester,
%      prof,
%      subtitle,
%      maketitle,
%      matrikelnummer,   
%      ^^A Add your command names here!
%    }
%    frame        = single,
%    backgroundcolor = \color{yellow!60},
%    framesep     = 5pt,
%    literate={Ö}{{\"O}}1 {Ä}{{\"A}}1 {Ü}{{\"U}}1 {ß}{{\ss}}1 {ü}{{\"u}}1 {ä}{{\"a}}1 {ö}{{\"o}}1 
%  }^^A
%}
%\newcommand*\LaTeXdemo@input{^^A
%  \MakePercentComment
%  \catcode`\^^M=10\relax
%  \small
%  \begingroup
%    \setkeys{lst}{
%      SelectCharTable=\lst@ReplaceInput{\^\^I}{\lst@ProcessTabulator}
%    }^^A
%    \leavevmode
%      \input{\jobname.tmp}^^A
%  \endgroup
%  \MakePercentIgnore
%}
%\LaTeXdemo@new{code and example}{^^A
%  \setbox\LaTeXdemo@box=\hbox\bgroup
%    \lst@BeginAlsoWriteFile{\jobname.tmp}^^A
%    \LaTeXdemo@common
%}{^^A
%    \lst@EndWriteFile
%  \egroup
%  \begin{center}
%    \ifdim\wd\LaTeXdemo@box>0.48\linewidth\relax
%      \hbox to\linewidth{\box\LaTeXdemo@box\hss}^^A
%        \begin{minipage}{\linewidth}
%          \LaTeXdemo@input
%        \end{minipage}
%    \else
%      \begin{minipage}{0.48\linewidth}
%        \LaTeXdemo@input
%      \end{minipage}
%      \hfill
%      \begin{minipage}{0.48\linewidth}
%        \hbox to\linewidth{\box\LaTeXdemo@box\hss}^^A
%      \end{minipage}
%    \fi
%  \end{center}
%}
%\LaTeXdemo@new{code only}{^^A
%  \LaTeXdemo@common
%}{^^A
%}
%
% \newinsert\bx@S
% \newinsert\bx@T
% \newinsert\bx@U
% \newinsert\bx@V
% \newinsert\bx@W
% \newinsert\bx@X
% \newinsert\bx@Y
% \newinsert\bx@Z
% \newinsert\bx@AA
% \newinsert\bx@BB
% \newinsert\bx@CC
% \newinsert\bx@DD
% \newinsert\bx@EE
% \newinsert\bx@FF
% \newinsert\bx@GG
% \newinsert\bx@HH
% \newinsert\bx@II
% \newinsert\bx@JJ
% \gdef\@freelist{\@elt\bx@A\@elt\bx@B\@elt\bx@C\@elt\bx@D\@elt\bx@E
%   \@elt\bx@F\@elt\bx@G\@elt\bx@H\@elt\bx@I\@elt\bx@J
%   \@elt\bx@K\@elt\bx@L\@elt\bx@M\@elt\bx@N
%   \@elt\bx@O\@elt\bx@P\@elt\bx@Q\@elt\bx@R
%   \@elt\bx@S\@elt\bx@T\@elt\bx@U\@elt\bx@V
%   \@elt\bx@W\@elt\bx@X\@elt\bx@Y\@elt\bx@Z
%   \@elt\bx@AA\@elt\bx@BB\@elt\bx@CC\@elt\bx@DD
%   \@elt\bx@EE\@elt\bx@FF\@elt\bx@GG\@elt\bx@HH
%   \@elt\bx@II\@elt\bx@JJ}
%
% \renewcommand*{\fps@table}{htb}
% \setlength\belowcaptionskip{10pt}
%
% \providecommand*\cls{\texttt}
% \providecommand*\env{\texttt}
% \providecommand*\file{\texttt}
% \providecommand*\opt{\texttt}
% \providecommand*\pkg{\textsf}
% \providecommand*\mac[1]{\texttt{\textbackslash #1}}
% \renewcommand*\thempfootnote{\@alph\c@mpfootnote}
% \renewcommand\@makefntext[1]%
%     {\noindent\makebox[.5em][l]{\@makefnmark\,}#1}
% \def\BibTeX{\rmfamily B\kern-.05em%
%    \ifx\testbx\f@series{\normalsize I\kern-.025em B\kern-.08em}%
%    \else{\textsc{i\kern-.025em b}\kern-.08em}%
%    \fi%
%    \unskip T\kern-.1667em\lower.7ex\hbox{E}\kern-.125emX}
% \makeatother
%
% \author{Martin Sievers\thanks{Vielen Dank an Agnieszka Oko\'nska für ihre fachliche Unterstützung}\\\href{mailto:martin.sievers@schoenerpublizieren.de?subject=hausarbeit-jura}{martin.sievers@schoenerpublizieren.de}}
% \title{Eine Klasse zur Erstellung juristischer Hausarbeiten an deutschen Universitäten}
% \date{Version \fileversion\ vom \filedate}
% \maketitle
%
% \section{Einleitung}
% \noindent Juristische Hausarbeiten zeichnen sich durch sehr strenge Formalia verbunden mit relativ wenig Flexibilität aus. Daher eignet sich \LaTeX{} hervorragend, um  solche Dokumente zu erstellen.
%
% Die Klasse \cls{hausarbeit-jura} richtet sich insbesondere an Erstsemester, die eine erste Hausarbeit erstellen müssen und dabei nicht automatisch zu den üblichen Textverarbeitungen greifen wollen.
%
% Die verwendeten Formalia orientieren sich dabei an konkreten Vorgaben an der Universität Trier, sollten allerdings auch an anderen Hochschulen grundsätzlich gültig sein.
%
% Für die Anwendung dieser Klasse werden grundsätzliche \LaTeX-Kenntnisse vorausgesetzt. Die Anleitung ersetzt ausdrücklich keine \LaTeX-Einführung.
%
% \section{Verwendete Pakete}
% \cls{hausarbeit-jura} erfindet das Rad nicht neu, sondern baut auf vorhandenen etablierten Klassen und Paketen auf. Grundlage ist die Klasse \cls{jurabook}, die u.\,a. die Gliederungsebenen zur Verfügung stellt. Neben einigen Standardpaketen wie \pkg{geometry} oder \pkg{babel} wird für die Literatur das Standardpaket \pkg{jurabib} mit passenden Einstellungen geladen.
%
% Zu allen Paketen kann man entsprechende Dokumentation auf CTAN finden.
%
% \section{Klassenoptionen}
% Um trotz aller Einheitlichkeit bestimmte Abweichungen komfortabel zu ermöglichen, bietet \cls{hausarbeit-jura} einige Klassenoptionen an.
%
% \subsection{Eingabekodierung}
% \DescribeOption{utf8, latin1, ansinew, applemac}
% Alle modernen \LaTeX-Editoren arbeiten mittlerweile mit der Standardzeichenkodierung \opt{utf8}. Wenn Sie eine andere Kodierung verwenden (müssen), können Sie diese als Klassenoption angeben. Die Einstellung wird an das Paket \pkg{inputenc} weitergereicht.
%
% \subsection{Verfügbare Sprachen}
% \DescribeOption{american, british, english}
% Als Standardsprache ist Deutsch eingestellt. Sie können zusätzlich Englisch bzw. britsches oder amerikanisches Englisch hinzunehmen. Dies geschieht über entsprechende Optionen, die an das Paket \pkg{babel} weitergereicht werden.
%
% Um auf eine dieser Sprache zu wechseln, nutzen Sie bitte die Standardmethoden wie \cs{selectlanguage}\marg{Sprache}.
%
% \subsection{Sonstige Optionen}%
% \DescribeOption{centerpagenumber}%
% Die Seitenzahlen werden standardmäßig rechts außen in der Fußzeile platziert. Durch die  Option \opt{centerpagenumber} erreichen Sie eine zentrierte Ausgabe.
%
%^^A\DescribeOption{texgyre}Um auf die moderneren Schriftvarianten aus der \TeX{}-Gyre-Familie zurückzugreifen, nutzen Sie bitte die Option \opt{texgyre}.
% \DescribeOption{oldfonts (ab v1.2)} Ab v1.2 sind die "`\TeX{} Gyre Fonts"' die Standardschriften. Um auf die "`alten"' Fonts zuzugreifen, 
% geben Sie bitte die Option \opt{oldfonts} an.
%
% \DescribeOption{nosuperscriptedition}
% Die Hochstellung der Auf"|lage lässt sich durch die Option \opt{nosuperscriptedition} verhindern.
%
% \DescribeOption{varioref}
% Möchten Sie für Referenzen auf das Paket \pkg{varioref} zurückgreifen, so können Sie dies bequem über die gleichnamige Klassenoption tun. Es werden einige Dinge entsprechend umdefiniert.
%
% \DescribeOption{headline (ab v1.2)}
% Sie können über die Option \opt{headline}\footnote{Vielen Dank an Tobias Hirning für den Hinweis.} %
% den Kurzautorenname sowie die Matrikelnummer immer oben links auf die Seiten drucken.
%
% \DescribeOption{headlinetitlepageleft\\(ab v1.3)}
% Die Autorenangaben auf der Titelseite werden standardmäßig rechtsbündig gesetzt. Mit der Option \opt{headlinetitlepageleft}\footnotemark[1] können sie dies in linksbündig ändern.
%
% \subsection{Satzspiegel}
% Die Klasse \cls{hausarbeit-jura} legt den Satzspiegel wie folgt fest:
% \begin{itemize}
%   \item Im Vorspann: \verb|left=2cm,right=2cm,top=2cm,bottom=2cm|
%   \item Im Hauptteil:\\\verb|left=7cm,right=1cm,top=1cm,bottom=1.5cm,footskip=0.5cm,marginparwidth=5cm,verbose|
% \end{itemize}
% Falls die Option \opt{headline} gesetzt wurde, wird auch die Option \opt{headline} für \pkg{geometry }gesetzt.
%
% \DescribeMacro{\setpgfront}\DescribeMacro{\setpgmain}Möchte man den Satzspiegel ändern, so sollte man hierfür die Befehle \cs{setpgfront\marg{Maße für Vorspann}} bzw. \cs{setpgmain\marg{Maße für Hauptteil}} nutzen.
%  
% \section{Aufbau eines Dokuments}
% \subsection{Texteingabe}
% Die Verwendung von Zeichen richtet sich grundsätzlich nach den Möglichkeiten, die \LaTeX{} bietet. Zusätzlich sind folgende Vorgaben zu beachten:
% \begin{itemize}
% \item Anführungszeichen sollten stets als \cs{enquote\marg{Text}} eingegeben werden, alternativ über 
% \lstinline+"+\`{}\ldots\lstinline+"'+.
% \item Auslassungspunkte werden mit \cs{dots\{\}} angegeben.
% \item Fußnoten werden in dem Makro \cs{footnote\marg{Text}} angegeben.
% \item Anmerkungen können als \cs{TODO\marg{Text}} integriert werden.
% \end{itemize} 
%
% \subsubsection{Querverweise}
% \cls{hausarbeit-jura} unterstützt das übliche Verfahren zur Nutzung von Querverweisen: Mit \cs{label\marg{Ankername}} wird dazu zunächst eine eindeutige Marke hinter dem Objekt gesetzt, auf das sie sich beziehen soll. Dann kann an beliebiger Stelle mit \cs{ref\marg{Ankername}} auf den Zähler genau dieser Marke zurückgegriffen werden.
%
% Die zugehörige Seitenzahl liefert \cs{pageref\marg{Ankername}}, auf eine Kombination beider Angaben kann mit \cs{xref\marg{Ankername}} zurückgegriffen werden.
%
% Darüber hinaus sorgt die Klassenoption \opt{varioref} für die Integration des gleichnamigen Pakets.
%
% Im Beispieldokument finden Sie entsprechende Beispiele.
% \clearpage
% \subsection{Titelseite}
% Für die Titelseite stehen folgende Makros zur Verfügung:
% \begin{center}
% \small   
% \begin{tabularx}{\textwidth}{Xl}
% \toprule
% Makro            & Voreinstellung \\\midrule
% \cs{title\marg{Art des Dokuments}} & Hausarbeit\\
% \cs{subtitle\marg{Titel der Übung}} & \emph{keine}\\
% \cs{author\oarg{Kurzautor}\marg{Autor}}   & \emph{keine}; mehrzeilige Eingabe mit \texttt{\textbackslash\textbackslash}\\
% \cs{matrikelnummer\marg{Matrikelnummer}} & \emph{keine}\\
% \cs{prof\marg{Betreuer}}     & \emph{keine}\\
% \cs{semester\oarg{Fachsemester}\marg{Semesterangabe}} & \emph{keine}\\
% \cs{date\marg{Abgabedatum}}     & Aktuelles Datum\\
% \bottomrule
% \end{tabularx}
% \end{center}
%
% \noindent Eine Beispieltitelseite könnte damit wie folgt aussehen:
% \begin{center}
% \begin{LaTeXdemo}
%   \documentclass{hausarbeit-jura}
%   ...
%   \begin{document}
%   % Angaben für die Titelseite
%   \title{Hausarbeit}
%   \subtitle{Übung für Anfänger im öffentlichen Recht}
%   \author{%
%      Otto Normalverbraucher\\
%      Musterweg 12\\
%      12345 Musterstadt\\
%      123456\\
%      3. Fachsemester}
%   \prof{Prof.\,Dr. X Y}
%   \semester{Wintersemester 2015/2016}
%   \date{15. Februar 2016}
%   \maketitle % Ausgabe der Titelseite
%   ... % Rest des Dokuments
%   \end{document}
% \end{LaTeXdemo}
%
% \DescribeMacro{ab v1.2}Ab Version 1.2 können Sie auch die neuen bzw. geänderten Makros nutzen.
% \begin{LaTeXdemo}
%   \documentclass{hausarbeit-jura}
%   ...
%   \begin{document}
%   % Angaben für die Titelseite
%   \title{Hausarbeit}
%   \subtitle{Übung für Anfänger im öffentlichen Recht}
%   \author{%
%      Otto Normalverbraucher\\
%      Musterweg 12\\
%      12345 Musterstadt
%   }
%   \matrikelnummer{123456}
%   \prof{Prof.\,Dr. X Y}
%   \semester[3. Fachsemester]{Wintersemester 2015/2016}
%   \date{15. Februar 2016}
%   \maketitle % Ausgabe der Titelseite
%   ... % Rest des Dokuments
%   \end{document}
% \end{LaTeXdemo}
% \end{center}
%
% \subsection{Verzeichnisse}
% \subsubsection{Inhaltsverzeichnis}
% Das Inhaltsverzeichnis wird wie üblich mit \cs{tableofcontents} an der gewünschten Stelle ausgegeben.
%
% \subsubsection{Literatur}
% Die Einbindung von Literatur geschieht in drei Schritten:
% \begin{enumerate}
% \item Erstellung einer Datenbank im bib-Format (z.\,B. mit Citavi)
% \item \DescribeMacro{\cite}Einfügung von Verweisen im Text mittels \cs{cite\oarg{Prenote}\oarg{Postnote}\marg{Schlüssel}}
%
% Je nach Literaturtyp werden die optionalen Argumente von \cs{cite} unterschiedlich verwendet:\\
% Aufsätze: \cs{cite\oarg{Anfangss., Belegs.}\marg{Schlüssel}}\\
% Kommentare: \cs{cite\oarg{Bearbeiter}\oarg{Stelle}\marg{Schlüssel}}\\
% Monographien: \cs{cite\oarg{Stelle}\marg{Schlüssel}}\\
% Festschriften: \cs{cite\oarg{FS Müller 100, 102}\marg{Schlüssel}}
% \item \DescribeMacro{\bibliography}Ausgabe des Quellenverzeichnisses an gewünschter Stelle:\\\cs{bibliography\marg{Quellensammlung}}
% \end{enumerate}
%
%
% \subsection{Gliederungsebenen}
% Es stehen die Gliederungsebenen folgende Gliederungsebenen zur Verfügung:\\ \cs{chapter\marg{Überschrift}}, \cs{section\marg{Überschrift}},\\ \cs{subsection\marg{Überschrift}}, \cs{subsubsection\marg{Überschrift}},\\ \cs{subthreesection\marg{Überschrift}}, \ldots, \cs{subeightsection\marg{Überschrift}},\\ \cs{paragraph\marg{Überschrift}}, \cs{subparagraph\marg{Überschrift}}.
%
% Zählweisen, Einzüge etc. aller Gliederungsebenen wurden entsprechend der mir bekannten Vorgaben angepasst.
% 
% \section{Ausblick}
% Gerne nehme ich Anregungen auf. Insbesondere bin ich an Vorgaben interessiert, die für juristische Hausarbeiten vorliegen, mit der aktuellen Klasse aber nicht umgesetzt werden können.
% \normalsize
% \StopEventually{%
%	\clearpage}
% ^^A	\selectlanguage{english}
% ^^A {\renewcommand\twocolumn[1][]{#1}
% ^^A  \PrintChanges\PrintIndex}}
% \section{Implementation}
%
%    \begin{macrocode}
%<*class>
%    \end{macrocode}
%    \begin{macrocode}
\NeedsTeXFormat{LaTeX2e}
\ProvidesClass{hausarbeit-jura}
[2016/04/25 v1.4 Eine Klasse für juristische Hausarbeiten (ao, ms)]
\RequirePackage{ifthen}
\DeclareOption{american}{%
    \PassOptionsToPackage{american,ngerman}{babel}
    \AtBeginDocument{\ClassInfo{hausarbeit-jura}{Die Sprache Englisch (Amerikanisch) wurde
      zusätzlich aktiviert.}}
}
\DeclareOption{british}{%
    \PassOptionsToPackage{british,ngerman}{babel}
    \AtBeginDocument{\ClassInfo{hausarbeit-jura}{Die Sprache Englisch (Britisch) wurde
      zusätzlich aktiviert}}
}
\DeclareOption{english}{%
    \PassOptionsToPackage{british,ngerman}{babel}
    \AtBeginDocument{\ClassInfo{hausarbeit-jura}{Die Sprache Englisch (Britisch) wurde
      zusätzlich aktiviert}}
}
\DeclareOption{latin1}{\PassOptionsToPackage{latin1}{inputenc}}
\DeclareOption{utf8}{\PassOptionsToPackage{utf8}{inputenc}}
\DeclareOption{ansinew}{\PassOptionsToPackage{ansinew}{inputenc}}
\DeclareOption{applemac}{\PassOptionsToPackage{applemac}{inputenc}}
\newif\if@centerpagenumber
\@centerpagenumberfalse
\DeclareOption{centerpagenumber}{\AtBeginDocument{\@classiclayoutfalse}}%\@centerpagenumbertrue}
\newif\if@texgyre
\@texgyretrue
\DeclareOption{texgyre}{\@texgyretrue}
\DeclareOption{oldfonts}{\@texgyrefalse}
\newif\if@superscriptedition
\@superscripteditiontrue
\DeclareOption{nosuperscriptedition}{\@superscripteditionfalse}
\newif\if@varioref
\@varioreffalse
\DeclareOption{varioref}{\@varioreftrue}
\newif\if@headline
\@headlinefalse
\DeclareOption{headline}{\@headlinetrue}
\newif\if@headlinetitlepageleft
\@headlinetitlepageleftfalse
\DeclareOption{headlinetitlepageleft}{\@headlinetitlepagelefttrue}
\ExecuteOptions{american,utf8}
\DeclareOption*{\PassOptionsToClass{\CurrentOption}{jurabook}}
\ProcessOptions\relax

\LoadClass[12pt,a4paper,oneside,showpagenumber,classiclayout]{jurabook}[2007/11/09 V0.91 (beta) juridical books (AS)]
\RequirePackage{ifluatex}
\RequirePackage{ifxetex}
\ifxetex
   \RequirePackage{fontspec}
   \setmainfont{Times New Roman}
\else
   \ifluatex
      \RequirePackage{fontspec}
      \setmainfont{Times New Roman}
   \else
      \RequirePackage{inputenc}
      \RequirePackage[T1]{fontenc}
      \if@texgyre
         \usepackage{tgtermes}
         \usepackage[matchlowercase]{tgheros}
         \usepackage[matchlowercase]{tgcursor}
      \else
         \usepackage{mathptmx}
         \usepackage[scaled=0.91]{helvet}
         \usepackage{courier}
      \fi%
      \RequirePackage{textcomp}
      \RequirePackage{eurosym}
      \let\texteuro\euro
      \DeclareInputText{128}{\,\euro}
   \fi%
\fi%
\RequirePackage{babel}
\RequirePackage{indentfirst}
\RequirePackage{geometry}
\if@headline
   \geometry{left=7cm,right=1cm,top=1cm,bottom=1.5cm,footskip=0.5cm,marginparwidth=5cm,verbose,includehead}%
\else
   \geometry{left=7cm,right=1cm,top=1cm,bottom=1.5cm,footskip=0.5cm,marginparwidth=5cm,verbose}%
\fi%
\savegeometry{pagemainmatter}
\newgeometry{left=2cm,right=2cm,top=2cm,bottom=2cm}
\savegeometry{pagefrontmatter}
\RequirePackage{ellipsis}
\RequirePackage[autostyle]{csquotes}
\defineshorthand{"`}{\openautoquote}
\defineshorthand{"'}{\closeautoquote}
\RequirePackage[final,tracking=smallcaps,expansion=alltext,protrusion=true]{microtype}%
\SetTracking{encoding=*,shape=sc}{50}%
\if@headline
   \renewcommand{\SetJuboPagestyle}{%
      \pagestyle{fancy}%
      \fancyhf{}%
      \lhead{\@shortauthor{} -- \@matrikelnummer}%
      \if@centerpagenumber
         \cfoot{\thepage}%
      \else
         \rfoot{\thepage}%
      \fi%
      \renewcommand{\headrulewidth}{0pt}%
   }%
\else
   \renewcommand{\headrulewidth}{0pt}%
   \fancypagestyle{plain}{%
      \fancyhead{}%
      \if@centerpagenumber\else
         \fancyhf{}%
         \fancyfootoffset[RO,RE]{2cm}%
         \rfoot{\thepage}%
      \fi}%
   \AtBeginDocument{%
      \fancyhead{}}%
\fi%

%JURABIB
\RequirePackage{jurabib}% Bibliografie
\jbedseplikecite% Die Separation der Herausgeber im Literaturverzeichnis entspricht der im Zitat
\renewcommand*{\bibatsep}{,}% Komma nach dem Titel des Buches oder des Artikels
\renewcommand*{\jbaensep}{,}% Komma nach der Auflage, vor dem Jahr
\renewcommand*{\bibbdsep}{}% kein Komma nach Ort und vor dem Jahr
\renewcommand*{\bibjtsep}{}% Komma vor dem Namen der Zeitschrift
\renewcommand*{\bibbtsep}{in:~}% für in collection, Komma vor dem Titel
\renewcommand*{\ajtsep}{}% Abstand nach dem Namen der Zeitschrift
\renewcommand*{\artnumberformat}{\unskip,\space H.~}% bei den Zeitschriften statt Nr. H.
\if@superscriptedition
   \jurabibsetup{%
      annotatorfirstsep=in,%
      bibformat=tabular,%
      titleformat={all,commasep},% Komma vor dem Titel
      commabeforerest,% Komma vor der Seitenangabe in der Fußnote
      superscriptedition=all,% Ausgabe oben
      howcited=all,%
      authorformat=dynamic%
   }%
\else
   \jurabibsetup{%
      annotatorfirstsep=in,%
      bibformat=tabular,%
      titleformat={all,commasep}, %Komma vor dem Titel
      commabeforerest, %Komma vor der Seitenangabe in der Fußnote
      howcited=all,%
      authorformat=dynamic%
      }%
\fi

\bibliographystyle{jurabib}
\bibtotoc

\renewcommand*{\bibleftcolumn}{\textwidth/3}%
\renewcommand*{\bibrightcolumn}{\textwidth-\bibleftcolumn-0.5cm}%

\if@varioref%
   \usepackage[ngerman]{varioref}
   \vref@addto\extrasngerman{%
      \def\reftextafter {auf S.~\thevpagerefnum}%
      \def\reftextbefore{auf S.~\thevpagerefnum}}
   \renewcommand{\xref}[1]{% vereinfacht Zitierungen
   \ref{#1} \vpageref{#1}}
\fi%

%%% Anpassung der Zähler
\renewcommand\thechapter{\Alph{chapter})}
\renewcommand\thesection{\Roman{section}.}
\renewcommand\thesubsection{\arabic{subsection}.}
\renewcommand\thesubsubsection{\alph{subsubsection})}
\renewcommand\thesubfoursection{(\arabic{subfoursection})}
\renewcommand\thesubfivesection{(\alph{subfivesection})}
\renewcommand\thesubsixsection{(\alph{subsixsection}\alph{subsixsection})}
\renewcommand\thesubsevensection{(\alph{subsevensection}\alph{subsevensection}\alph{subsevensection})}
\renewcommand\thesubeightsection{(\roman{subeightsection})}

\renewcommand{\jubo@settocindents}{%
  \if@chapterprefix%
    \settowidth{\chapternumwidth}{\chaptername~999\chapterextra\space}% warum drei x 9? zweimal ist zu knapp
  \else% chaptersuffix:
    \settowidth{\chapternumwidth}{999.~\chaptername\chapterextra\space}%
  \fi%
  \settowidth{\sectionnumwidth}{III.\ }%
  \settowidth{\subsectionnumwidth}{8.\ }%
  \settowidth{\subsubsectionnumwidth}{b)\ }%
  \settowidth{\subthreesectionnumwidth}{bb)\ }%
  \settowidth{\subfoursectionnumwidth}{(8)\ }%
  \settowidth{\subfivesectionnumwidth}{(b)\ }%
  \settowidth{\subsixsectionnumwidth}{(bb)\ }%
  \settowidth{\subsevensectionnumwidth}{(bbb)\ }%
  \settowidth{\subeightsectionnumwidth}{(iii)\ }%
  \if@chapterprefix%
    \setlength{\sectiontocindent}{\chapternumwidth}%
  \else%
    \settowidth{\@tempdima}{\bfseries 99.\space}%
    \setlength{\sectiontocindent}{\@tempdima}%
  \fi
  \if@fixtocindent%
    \setlength{\sectiontocindent}{\jb@defaulttocindent@length}% = 1.5em per default
  \fi%
  \setlength{\subsectiontocindent}{\sectiontocindent}%
  \addtolength{\subsectiontocindent}{\sectionnumwidth}%
  \setlength{\subsubsectiontocindent}{\subsectiontocindent}%
  \addtolength{\subsubsectiontocindent}{\subsectionnumwidth}%
  \setlength{\subthreesectiontocindent}{\subsubsectiontocindent}%
  \addtolength{\subthreesectiontocindent}{\subsubsectionnumwidth}%
  \setlength{\subfoursectiontocindent}{\subthreesectiontocindent}%
  \addtolength{\subfoursectiontocindent}{\subthreesectionnumwidth}%
  \setlength{\subfivesectiontocindent}{\subfoursectiontocindent}%
  \addtolength{\subfivesectiontocindent}{\subfoursectionnumwidth}%
  \setlength{\subsixsectiontocindent}{\subfivesectiontocindent}%
  \addtolength{\subsixsectiontocindent}{\subfivesectionnumwidth}%
  \setlength{\subsevensectiontocindent}{\subsixsectiontocindent}%
  \addtolength{\subsevensectiontocindent}{\subsixsectionnumwidth}%
  \setlength{\subeightsectiontocindent}{\subsevensectiontocindent}%
  \addtolength{\subeightsectiontocindent}{\subsevensectionnumwidth}%
  \setlength{\paragraphtocindent}{\subeightsectiontocindent}%
  \addtolength{\paragraphtocindent}{\subeightsectionnumwidth}%
  \setlength{\subparagraphtocindent}{\paragraphtocindent}%
  \addtolength{\subparagraphtocindent}{\jb@defaulttocindent@length}%
}

%%% Kapitel anpassen
\renewcommand{\jb@chapter@style}{\LARGE\bfseries}
\setlength{\jb@chapterafter@length}{20\p@}
\setlength{\jb@chapterbefore@length}{15\p@}
%%%
%%% Abschnitte anpassen
\newlength{\sectionbefore}
\setlength{\sectionbefore}{-2ex}
\newlength{\sectionafter}
\setlength{\sectionafter}{1ex}
%
\renewcommand\section{\@startsection%
  {section}%
  {1}%
  {\z@}%
  {\sectionbefore}%
  {\sectionafter}%
  {\normalfont\Large\bfseries}}
\renewcommand\subsection{\@startsection%
  {subsection}%
  {2}%
  {\z@}%
  {\sectionbefore}%
  {\sectionafter}%
  {\normalfont\large\bfseries}}
\renewcommand{\subsubsection}{\@startsection%
   {subsubsection}%
   {3}%
   {\z@}%
   {\sectionbefore}%
   {\sectionafter}%
   {\normalfont\normalsize\bfseries}}
\renewcommand{\subthreesection}{\@startsection%
  {subthreesection}%
  {4}%
  {\z@}%
  {\sectionbefore}%
  {\sectionafter}%
  {\normalfont\normalsize\bfseries}}
\let\subsubsubsection\subthreesection
\renewcommand*{\subfoursection}{\@startsection%
  {subfoursection}%
  {5}%
  {\z@}%
  {\sectionbefore}%
  {\sectionafter}%
  {\normalfont\normalsize\bfseries}}
\let\subsubsubsubsection\subfoursection
\renewcommand*{\subfivesection}{\@startsection%
  {subfivesection}%
  {6}%
  {\z@}%
  {\sectionbefore}%
  {\sectionafter}%
  {\normalfont\normalsize\bfseries}}
\let\subsubsubsubsubsection\subfivesection
\renewcommand*{\subsixsection}{\@startsection%
  {subsixsection}%
  {7}%
  {\z@}%
  {\sectionbefore}%
  {\sectionafter}%
  {\normalfont\normalsize\bfseries}}
\let\subsubsubsubsubsubsection\subsixsection
\renewcommand*{\subsevensection}{\@startsection%
  {subsevensection}%
  {8}%
  {\z@}%
  {\sectionbefore}%
  {\sectionafter}%
  {\normalfont\normalsize\bfseries}}
\let\subsubsubsubsubsubsubsection\subsevensection
\renewcommand*{\subeightsection}{\@startsection%
  {subeightsection}%
  {9}%
  {\z@}%
  {\sectionbefore}%
  {\sectionafter}%
  {\normalfont\normalsize\bfseries}}
\let\subsubsubsubsubsubsubsubsection\subeightsection
%%%
\let\@ORIGfrontmatter\frontmatter
\renewcommand\frontmatter{\@ORIGfrontmatter%
	\loadgeometry{pagefrontmatter}%
	\onehalfspacing% muss nochmal gesetzt werden, da es sonst durch loadgeometry verloren geht
	\reversemarginpar% muss nochmal gesetzt werden, da es sonst durch loadgeometry verloren geht
   \fancyfootoffset[RO]{0cm}%
   \pagestyle{fancyplain}%
   \if@headline\fancyheadoffset[L]{0pt}\fi}
\AtBeginDocument{\frontmatter}	

\let\@ORIGmainmatter\mainmatter
\renewcommand\mainmatter{\@ORIGmainmatter%
   \loadgeometry{pagemainmatter}%
   \onehalfspacing%
   \reversemarginpar%
	\fancyfootoffset[RO,RE]{0cm}%
   \if@headline\fancyheadoffset[L]{\marginparsep+\marginparwidth}\fi}
\renewcommand{\chapter}{%
  \ifnum\value{chapter}=0\mainmatter\fi% Automatisch auf Hauptteil umschalten
  \chapterlevel%
  \if@headline\else\jubo@thispagestyle@empty\fi%
  \global\@topnum\z@%
  \@afterindentfalse
  \secdef\@chapter\@schapter%
}

\newcommand{\setpg}[2][]{\setpgfront{#1}\setpgmain{#2}}
\newcommand{\setpgmain}[1]{\newgeometry{#1}\savegeometry{pagemainmatter}}
\newcommand{\setpgfront}[1]{\newgeometry{#1}\savegeometry{pagefrontmatter}}


%%% Kein § vor der Kapitelnummer; korrekte Abstände
\renewcommand{\make@jubo@chaptercomplete}{%
  \if@chapterprefix%
    \renewcommand{\jubo@chaptercomplete}{\thechapter\chapterextra}
  \else%
    \renewcommand{\jubo@chaptercomplete}{\thechapter\chapterextra}
  \fi%
  }

%%% Eintrag der Kapitel ohne Nummer im TOC auch fett
\renewcommand{\jb@tocextras@style}{\normalfont\bfseries}

%%% Autor in Normalgröße
\renewcommand*{\jb@author@style}{\normalsize}

%%% Voreinstellung für Titelseite
\def\@title{Hausarbeit}

\def\@shortauthor{\@empty}
\let\@@@author\author
\def\author{\@ifnextchar[{\@@author}{\@@@author}}
\def\@@author[#1]#2{\gdef\@shortauthor{#1}\@@@author{#2}}
%%% Neue Befehle für die Titelseite
\def\@semester{}
\def\@fachsemester{}
\def\semester{\@ifnextchar[{\@@semester}{\@@@semester}}
\def\@@semester[#1]#2{\gdef\@fachsemester{#1}\gdef\@semester{#2}}
\def\@@@semester#1{\gdef\@semester{#1}}

\def\@prof{}
\def\prof#1{\gdef\@prof{#1}}

\def\@matrikelnummer{}
\def\matrikelnummer#1{\gdef\@matrikelnummer{#1}}

\renewcommand{\maketitle}{%
  \begin{titlepage}%
    \let\footnotesize\small
    \let\footnoterule\relax
    \let\footnote\thanks
    {\jb@author@style
       \if@headlinetitlepageleft\else\hfill\fi
       \begin{tabular}[t]{l}%
         \@author%
         \ifx\@matrikelnummer\@empty\else\tabularnewline\@matrikelnummer\fi%
         \ifx\@fachsemester\@empty\else\tabularnewline\@fachsemester\fi%
       \end{tabular}\par%
      }
    \null\vfil%
    %\vskip 60\p@
    \begin{center}%
      {\jb@title@style\vskip 2ex \@title\par}%
      \ifx\@subtitle\@empty% \ifthenelse funzt hier nicht wg. moegl. '\\'
      \else
        {\jb@subtitle@style\vskip 4ex\@subtitle \par}%
      \fi%
      \vskip 1ex%
      {\ifx\@prof\@empty%
       \else\jb@subtitle@style Bei\ \@prof \par\fi}
       \vskip 2ex
      {\jb@subtitle@style \@semester}
      \vskip 4ex
      {\jb@titledate@style\@date\par}%
    \end{center}\par
    %{\jb@thanks@style\@thanks\par}
    \vfil\null%
    \@emptytitles
  \end{titlepage}%
  \setcounter{page}{2} %%% damit die Seitenzählung korrekt ist
}%

\renewcommand{\@emptytitles}{%
  \setcounter{footnote}{0}%
  \global\let\thanks\relax
  \global\let\maketitle\relax
  \global\let\@thanks\@empty
  \global\let\@author\@empty
  \global\let\@date\@empty
  \global\let\@title\@empty
  \global\let\title\relax
  \global\let\author\relax
  \global\let\date\relax
  \global\let\and\relax
  \global\let\@subtitle\@empty
  \global\let\@publishinfo\@empty
  \global\let\@birthplace\@empty
  \global\let\@publishplace\@empty
  \global\let\@publishyear\@empty
  \global\let\@firstexaminer\@empty
  \global\let\@secondexaminer\@empty
  \global\let\@examdate\@empty
  \global\let\subtitle\relax
  \global\let\publishinfo\relax
  \global\let\birthplace\relax
  \global\let\publishplace\relax
  \global\let\publishyear\relax
  \global\let\firstexaminer\relax
  \global\let\secondexaminer\relax
  \global\let\examdate\relax%
  \global\let\@semester\@empty
  \global\let\@prof\@empty
  \global\let\semester\relax
  \global\let\prof\relax
}

\def\@setsize#1#2#3#4{%
  % Modified 1993.04.07--GDG per KPC
  \@nomath#1%
  \let\@currsize#1%
  \baselineskip #2%
  \baselineskip \baselinestretch\baselineskip
  \parskip \baselinestretch\parskip
  \setbox\strutbox \hbox{%
    \vrule height.7\baselineskip
           depth.3\baselineskip
           width\z@}%
  \skip\footins \baselinestretch\skip\footins
  \normalbaselineskip\baselineskip#3#4}

\let\@ORIGgeometry\geometry
\renewcommand\geometry[1]{\@ORIGgeometry{#1}%
    \setlength{\marginparwidth}{.6\oddsidemargin}%
    \addtolength{\marginparwidth}{1in}}

\let\frqq\glqq
\let\flqq\grqq
\tolerance 1414
\hbadness 1414
\emergencystretch 1.5em
\hfuzz 0.3pt
\clubpenalty=5000
\widowpenalty=9999
\vfuzz \hfuzz
\raggedbottom
%    \end{macrocode}
%    \begin{macrocode}
%</class>
%    \end{macrocode}
%
% \Finale
%
% \iffalse
%
%<*bib>
@ARTICLE{hau98,
  author = {Wolfgang Hau},
  title = {Anmerkung zu BGH, Urt. vom 1.10.1997},
  journal = {JA},
  year = {1998},
  pages = {358--361}
}

@BOOK{rosenberg10,
  title = {Zivilprozessrecht},
  year = {2010},
  author = {Leo Rosenberg and Karl Heinz Schwab and Peter Gottwald},
  shortauthor = {Rosenberg/Schwab/Gottwald},
  shorttitle = {ZPO},
  address = {München},
  edition = {17}
}

@BOOK{prutting10,
  title = {ZPO Kommentar},
  year = {2010},
  editor = {Hanns Prütting and Markus Gehrlein},
  shortauthor = {Prütting/Gehrlein},
  shorttitle = {ZPO},
  address = {Köln},
  edition = {1}
}
%</bib>
%<*demo>
%% !TeX program = pdflatex
%% !TeX TXS-program:bibliography = txs:///bibtex8
\documentclass{hausarbeit-jura}
\usepackage{blindtext}

\begin{document}
%%% Vorspann mit Titel, Sachverhalt und den Verzeichnissen
\title{Hausarbeit}
\subtitle{Übung für Anfänger im öffentlichen Recht}
% \author{Otto Normalverbraucher\\Musterweg 12\\12345 Musterstadt\\123456\\3. Fachsemester}
\author{Otto Normalverbraucher\\Musterweg 12\\12345 Musterstadt}
\matrikelnummer{123456}
\semester[3. Fachsemester]{Wintersemester 2015/2016}
\prof{Prof.\,Dr. X. Y.}
\date{15. Februar 2016}
\maketitle

\tableofcontents %Inhaltsverzeichnis
\bibliography{hausarbeit-demo}
%%%%%%%%%%%%%%%%%%%%%%%%%%%%%%%%%%%%%%%%%%%%%%%%%%%%%%%%%%%%%%%%%%%%%%%%%%%%%
% ab hier Hauptteil
\chapter{Ein erstes Kapitel}\label{chap:ErstesKapitel}
\TODO{Da fehlt noch was!}

\section{Abschnitt}
\blindtext[4]

Eine erste Fußnote\footnote{Dies ist eine Fußnote mit Zitat \cite[348, 349]{hau98}.}

\section{Noch ein Abschnitt}
\blindtext

\subsection{Ein Unterabschnitt}
\blindtext

\subsubsection{Ein Unterabschnitt}
\blindtext

\subthreesection{Ein Unterabschnitt}
\blindtext

\subfoursection{Ein Unterabschnitt}
\blindtext

\subfivesection{Ein Unterabschnitt}
\blindtext

\subsixsection{Ein Unterabschnitt}
\blindtext

\subsevensection{Ein Unterabschnitt}
\blindtext

\subeightsection{Ein Unterabschnitt}
\blindtext

\chapter{Noch ein Kapitel}
Wie in Kapitel~\ref{chap:ErstesKapitel} auf S.~\pageref{chap:ErstesKapitel} \ldots

Wie in Kapitel~\xref{chap:ErstesKapitel} \ldots

\end{document}
%</demo>
%\fi 